\section{Introduction}
As this thesis is primarily an exploration of epitaxial interfaces using a wide array of experimental techniques, the background material here will cover the theoretical concepts that are regularly drawn upon to present the growth models presented elsewhere. Models presented later in the document are primarily of a conceptual nature and as such background here will also be presented in that manner. Mathematics will be used where applicable but will be generally avoided, as the restricted assumptions are of little use to the work presented later.

There are a few key theoretical concepts which must be understood in order to best describe the expanded epitaxial model proposed here. The first, and most integral to the discussion is an examination of the model of epitaxy as currently defined in the literature. This is integral to differentiate where the material systems examined diverge from the idealized models presented in the literature. As part of the examination of epitaxy, particular attention will be given to nucleation and clusters on surfaces as this examination hinges on the role of the epitaxial interface. Beyond epitaxy, the other key subjects which require background exposition are surface reconstructions, which describe the properties of the other side of the epitaxial interface, the substrate, and atomic bonding, the connections which reach across the epitaxial interface.

\section{Epitaxy}
\subsection{Homoepitaxy}
Equilibium versus kinetics. Even the most perfectly matched systems in terms of atomic bonding and lattice match can grow with defects if the deposition rate is too fast.

\subsection{Heteroepitaxy}
SiGe as the model system.



III-V lattice matched as next best model system
\subsubsection{Strain}
\begin{figure}
    \centering
    \missingfigure{Diagram of matched, strained and relaxed films}
    \caption{\label{fig:back_strain}Matched, strained and relaxed films}
\end{figure}
\begin{equation}
f = \frac{a_f - a_s}{a_s}
\end{equation}
* Matched
* Strained
* Relaxed films

\subsection{Clusters and Nucleation}
\begin{figure}
    \centering
    \missingfigure{SK, layer-by-layer, and van der mere}
    \caption{\label{fig:back_growth_mode}SK, layer-by-layer and frank van der mere growth modes}
\end{figure}

\begin{figure}
    \centering
    \missingfigure{pg 383 regions of types of growth}
    \caption{\label{fig:back_nucleation_regions}Figure from Ohring page 383}
\end{figure}

\section{Surface Reconstructions} 
If one were to consider a crystal of infinite size, and subsequently cut it along any direction, the resulting surface would be a very high energy surface. Bonds will be completely unsatisfied (truncated electron clouds) and the surrounding environment for each surface atom will be significantly different than they were in the bulk. This high energy condition is clearly not the equilibrium state and must be resolved. The resolution of this high energy state that results when surfaces of a single crystal are exposed is called the surface reconstruction. While the name surface reconstruction implies changes to just the surface atoms, the surface reconstruction can extend a number of layers into crystal. Thus, the changes that occur at the surface of a crystal can result in a very different interface than the bulk crystal for the purposes of epitaxy.

The atoms on the surface of a newly cleaved crystal have a number of options for resolving their high energy configuration. The simplest of options for atoms in those high energy states to reduce their energy is to move, that is, to change it's position relative to other atoms. These movements happen both the in-plane and out-of-plane directions. Both atomic translations can cause additional shifts in the atomic layers below the surface.  Both in-plane and out-of-plane atomic movement breaks the bulk symmetry of the original crystal, resulting in a surface with lower symmetry. A schematic example of the atom movement for surface reconstruction is shown in \cref{fig:back_recon_move}.
\begin{figure}
    \centering
    \missingfigure{Image of before-after atom moving surface reconstruction}
    \caption{\label{fig:back_recon_move}Example of surface reconstruction with atomic movement}
\end{figure}

A more complex response resulting in a surface reconstruction that can arise is a change in the actual bonding in the upper layers of the crystal. Dimers and higher order bonding can form at the surface of the crystal, partially resolving the dangling bonds present on the surface. This dimerization of the surface can significantly reduce the chemical activity of the surface since the electronic structure has been satisfied internally to the crystal. An example of such dimerization is shown in \cref{fig:back_recon_dimer}
\begin{figure}
    \centering
    \missingfigure{Dimer formation during surface reconstruction}
    \caption{\label{fig:back_recon_dimer}Dimerization on surface reconstruction}
\end{figure}

\subsection{Stability, Thermodynamics and Kinetics of Surface Reconstructions}

\section{Atomic Bonding}
\begin{figure}
    \centering
    \missingfigure{Ohring page 374 - potential energy versus distance}
    \caption{\label{fig:back_bond_potential}Potential energy versus distance for atom respect to surface}
\end{figure}

