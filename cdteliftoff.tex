\section{Introduction}
The growth of CdTe on oxide substrates, specifically single crystal sapphire, has been a region of great interest for the Preston research group.
Work has been published on optimization of growth parameters, the role of lattice constants, and the optical properties of the resulting thin films\cite{Neretina2009a,Neretina2008b,Neretina2009b,Neretina2007,Neretina2006,cdte-optical}.

While the work investigating the growth parameters and properties of the resulting thin films has progressed to a high level, less investigation has been focused on explaining the surprising success that has been achieved with this material system.
With a relative lattice mismatch of 3.65\% between CdTe and sapphire, a difference in the 
crystallographic space group (cubic CdTe versus hexagonal sapphire), and chemical differences (high ionicity semiconductor CdTe versus complex oxide sapphire), the high quality single crystal nature of the grown thin films is far from expected.

In this work investigating the CdTe on sapphire heteroepitaxial material system, the unexpected high quality growth is examined through the lens of symmetry and energy at the epitaxial interface.
These examinations support an explanatory model of the epitaxial alignment and defects present in CdTe thin films.
These examinations also reveal the previously undocumented and highly surprising result that CdTe is not bonded nearly as strongly as expected to the sapphire substrates after epitaxial growth, resulting in the technologically relevant liftoff phenomenon.
The liftoff phenomenon is examined and its resulting freestanding thin films are characterized.
This liftoff phenomenon has been discovered to be robust enough to apply for a provisional patent\cite{patent} which has since been converted into a full simultaneous worldwide patent application.
This work was completed in collaboration with Mr.~Stephen M. Jovanovic (growths), Dr.
Kristoffer Mienander (surfaces), Ms.~Steffi Woo (TEM), and Dr. Martin Couillard (STEM) and relies on prior work by Dr.~Robert Hughes and Dr.~Svetlana Neretina.
\section{Background}
CdTe is a cubic semiconductor (a = 6.483\AA{}) with a strong propensity to grow (111)-up, that is alternating layers of cadmium and tellurium, regardless of the structure of the underlying substrate.
Thus the key requirement for the growth of quality single crystal thin films is to control the nucleation and in-plane orientation. \textalpha-Al\(_2\)O\(_3\) (sapphire) is a rhobehedral complex oxide (a = 4.758~\AA{}, c = 12.991~\AA{}), which presents a hexagonal surface net on its c-plane surface.
As had been previously investigated by the Preston research group, the (110) diagonal of CdTe matches to double of the lattice constant of sapphire to within 3.65\%, providing a geometric template for the epitaxial alignment of (111)-up on the 
c-plane surface.
While the c-plane sapphire offers an epitaxial template for the CdTe, the mismatch of cubic on hexagonal symmetries offers two equivalent orientations for the CdTe crystal, as shown in \cref{fig:cdteliftoff_geometry}.
\begin{figure}
 \centering \includegraphics{cdteliftoff_geometry}
 \caption{\label{fig:cdteliftoff_geometry}Geometric model of cubic CdTe crystal structure fit on hexagonal c-plane sapphire surface.}
\end{figure}
Despite the geometric equivalence of two orientations of cubic on hexagonal symmetry, growths done by this research group have resulted in a single orientation of CdTe on the sapphire surface.
Previous work by this research group attempted to explain the preferred orientation through experiments in the modification of sapphire\cite{Neretina2009b}, suggesting that the energy considerations at the surface play a key factor in epitaxy.
\section{Experimental}
CdTe thin films were deposited on single crystal c-plane \textalpha-Al\(_2\)O\(_3\) \(\pm\) 0.5\degree{} wafers, obtained from MTI Crystals Inc and diced into 12~mm \(\times\) 12 mm squares.
Prior to deposition, substrates were solvent cleaned in an ultrasonic bath.
Samples were loaded into a custom pulsed laser deposition chamber at a base pressure of \(1\cdot10^{-7}\)~Torr and in-situ annealed at 450\celsius{} for 30~minutes.
CdTe thin films were deposited by pulsed laser deposition using a GSI Lumonics IPEX-848 KrF excimer laser with a wavelength of 248 nm.
Pulses from the laser were focused and rastered radially onto a rotating CdTe 2.54 cm diameter target with a spot size of 4.25 mm\textsuperscript{2} and average energy density of 1.8 J/cm\textsuperscript{2}.
The CdTe 5N (99.999\%) pressed powder 
target obtained from Princeton Scientific was stoichiometric and undoped.
During growth samples were kept at a nominal temperature of 300\celsius{} via a Pt-Rh thermocouple on the growth furnace surface.
Films were grown to a thickness of 100 nm, as determined by optical and stylus profilometry.

Structural information was obtained using 2DXRD and TEM/STEM techniques.
A Bru\-ker SMART\-6000 CCD detector on a Bruker 3-circle D8 goniometer with Rigaku RU-200 rotating anode X-ray generator and parallel-focusing mirror optics were used for the data collection.
2DXRD data was processed into pole figures using Bruker GADDS\@.
TEM/STEM preparation was completed by FIB preparation of cross sections followed by gentle Ar ion milling.
TEM imaging was collected using a Philips CM-12 and STEM imaging was collected using a FEI Titan 80-300HB with aberration correction.
\section{Results}
CdTe thin films had been previously thoroughly structurally characterized via 2DXRD, AFM and TEM\@. As such, one of the next steps to fully characterize the CdTe thin films was to produce larger uniform films\cite{stephen-thesis} and to electrically characterize these films via resistivity and the Hall effect.
As part of this investigation, lithographic patterning of Pt contacts was performed to create van der Pauw geometry.
Upon performing an acetone soak, as the photoresist dissolved, and the metal film floated off the sample, it remained in one piece, lifting off the areas of CdTe that were in contact with the metal from the epitaxial substrate.
For a sample that had been previously characterized to be single crystal, this result was unexpected, as all that was required for this liftoff was an ultrasonic bath.
\begin{figure}
 \centering \includegraphics[width=0.8\textwidth]{cdteliftoff_nanowires}
 \caption[Toppled CdTe nanowire]{\label{fig:cdteliftoff_nanowires}Toppled CdTe nanowire, the bare area where the substrate was not coated by gold for SEM examination is clearly visible.}
\end{figure}
This weak bonding phenomenon had been previously hinted at when CdTe nanowires (which are discussed in more detail in \cref{sec:cdtenanowires}) were observed in SEM to have toppled in place, as shown in \cref{fig:cdteliftoff_nanowires}, an event that could not have happened unless the bond strength with the interface was weak.

After the discovery of the liftoff phenomenon with lithographic patterning, simpler methods of liftoff were attempted.
Strong adhesive tapes were found to successfully remove thin films with a simple mechanical peeling motion.
While the tape peeling was effective, the large curvatures caused cleaving and breakage in the lifted off film.
Adhesive epoxies were attempted and found to provide a more rigid carrier, eliminating cleaving and breakage.
Yields of liftoff are highly dependent on the quality of bonding to the CdTe surface, clean surfaces and effective adhesives are key.
Numerous other bonding methods were tested including optical element adhesive, polymer films and the simplification of liftoff by the addition of liquid N\textsubscript{2} as thermal shock.
The generalized adhesive based liftoff process is shown in \cref{fig:cdteliftoff_process}
\begin{figure}
 \centering \includegraphics[width=\textwidth]{cdteliftoff_process}
 \caption{\label{fig:cdteliftoff_process}Generalized CdTe liftoff process.}
\end{figure}
The flexible non-conductive liftoff carriers offered a first step to producing freestanding thin films, but electrical contact to the films is a key property in order to yield devices.
To this end, thin films were coated with metal, first platinum (a known CdTe contact material) and then copper, to create a reactive but temperature stable surface.
Samples were then wetted with solder paste and placed metal side down onto a copper surface, and thermally cycled through a solder reflow curve, as shown in \cref{fig:cdteliftoff_process} step 4b.
Upon removal from the oven, the films were found to have bonded to the copper surface and spontaneously lifted off from the original sapphire substrate.
These experiments have demonstrated that the production of freestanding thin films by the liftoff phenomenon is simple and straightforward.

Concurrent to investigations into the production of freestanding thin films via the liftoff process, the question arose as to the properties of these films when compared to those attached to the epitaxial substrate.
2DXRD measurements were undertaken on thin films before, \cref{fig:cdteliftoff_F22_attached} and after \cref{fig:cdteliftoff_F22_released}, liftoff processing using two part epoxy.
\begin{figure}
 \centering
 \begin{subfigure}[b]{0.45\textwidth}
  \centering \includegraphics[width=\textwidth]{cdteliftoff_F22_attached}
  \caption{\label{fig:cdteliftoff_F22_attached}(111) Pole figure of as-grown CdTe on sapphire.}
 \end{subfigure}\quad%
 \begin{subfigure}[b]{0.45\textwidth}
  \centering \includegraphics[width=\textwidth]{cdteliftoff_F22_released}
  \caption{\label{fig:cdteliftoff_F22_released}2DXRD of single crystal CdTe thin film on epoxy carrier.}
 \end{subfigure}%
 \caption{\label{fig:cdteliftoff_2DXRD}Attached and released CdTe pole figures, area of scaled intensity shows secondary phases present from twinning.}
\end{figure}
2DXRD measurements of lifted off films, when compared to as-grown films, show a qualitatively identical pole figure.
Peak broadness is unchanged, only the bleed through of the sapphire (024) peak disappears after the removal of the sapphire substrate.
The change in the magnitude of the secondary twinning phase is due to the change in location of the two 2DXRD measurements on the sample.

\subsection{Epitaxial Interface Characterization}
The liftoff phenomenon observed hinges on the intimate relationship between CdTe and sapphire.
In order to better understand the nature of the of this interface, experimental and computational investigations of the interface were undertaken.
Samples were aligned using 2DXRD in order to prepare (01\(\overline{1}\)) cross-sections of CdTe using FIB for STEM imaging.
Imaging was initially intended to provide atomic scale resolution of the interface, however it was found to be blurry for all samples prepared.
It is unclear if this blurriness was an intrinsic property of the epitaxy process or as a result of sample preparation of the mechanically dissimilar interface.

Due to the atomic contrast available from STEM imaging, alignment of the Cd-Te dumbbells, along with absolute alignment of the physical STEM sample, allowed for determination of the polarity of the CdTe thin film.
STEM imaging of the (01\(\overline{1}\)) cross-section of CdTe was aligned from the FIB attachment point as in \cref{fig:cdteliftoff_stem_low}.
After alignment, high magnification STEM was taken at the interface as in \cref{fig:cdteliftoff_stem_high}, showing distinct \{110\} family dumbbells.
STEM image intensity was integrated along the dumbbells as plotted in \cref{fig:cdteliftoff_stem_intensity}, the modulation in the intensity data indicates a Te-Cd dumbbell oriented away from the interface.
The STEM intensity modulation, combined 2DXRD crystal alignment and alignment of the STEM lamella, indicates the polarity of the thin film is CdTe(111)A, i.e. the film presents Cd-single bond up.
The model of the CdTe(111)A crystal observed is shown in \cref{fig:cdteliftoff_stem_model}.
\begin{figure}
    \centering
    \begin{subfigure}[t]{0.47\textwidth}
        \centering \includegraphics[width=\textwidth]{cdteliftoff_stem_low}
        \caption{\label{fig:cdteliftoff_stem_low}Low magnification STEM of CdTe on sapphire, by FIB preparation, showing attachment point.}
    \end{subfigure}\quad%
    \begin{subfigure}[t]{0.47\textwidth}
        \centering \includegraphics[width=\textwidth]{cdteliftoff_stem_high}
        \caption{\label{fig:cdteliftoff_stem_high}High magnification STEM of CdTe on sapphire showing interface, CdTe dumbbells highlighted where intensity integration was performed.}
    \end{subfigure}%
    \caption{\label{fig:cdteliftoff_stem}CdTe on sapphire STEM images used for polarity determination.}
\end{figure}

\begin{figure}
    \centering
    \begin{subfigure}[t]{0.50\textwidth}
        \centering \includegraphics[width=\textwidth]{cdteliftoff_stem_intensity}
        \caption{\label{fig:cdteliftoff_stem_intensity}Intensity plot of STEM image from \cref{fig:cdteliftoff_stem_high}, intensity is higher-lower away from surface, indicating Te-Cd dumbbells}
    \end{subfigure}\quad%
    \begin{subfigure}[t]{0.45\textwidth}
        \centering \includegraphics[width=\textwidth]{cdteliftoff_stem_model}
        \caption{\label{fig:cdteliftoff_stem_model}Atomic model corresponding to CdTe(111)A, Cd atoms in beige, Te atoms in orange}
    \end{subfigure}%
    \caption{\label{fig:cdteliftoff_stem2}CdTe on sapphire STEM images used for polarity determination.}
\end{figure}
%
%While STEM imaging was unable to determine the atomic configuration of the interface,  some information can be determined about the interface from simulation of the interaction of adatoms of Cd and Te with the sapphire. Density functional theory (DFT) modelling was undertaken using ABINIT\cite{Gonze20092582}, to determine the potential energy landscape of Cd and Te adatoms. The stable sapphire (0001) surface was used as the starting configuration and Cd and Te atoms were scanned over the sapphire unit cell in a \(11 \times 11\) grid to compute the potential energy at each point. The potential energy map for Cd and Te on the bare sapphire surface is shown in \cref{fig:cdteliftoff_dft1} corresponding to the unit cell shown in \cref{fig:cdteliftoff_sapphire_unit}. From these potential energy maps, the lowest energy configuration is expected to be a Te atom located on one of the oxygen positions. Once a second atom is added to the configuration, as in \cref{fig:cdteliftoff_dft2}, the lowest energy configuration is the Te placed on the Al\textsubscript{1} position and the Cd atom placed on the Al\textsubscript{3} position.

\section{Discussion}
The mechanism by which liftoff of CdTe on sapphire substrates occurs has not been definitively determined.
A number of theoretical models have been examined by this group to explain the exceptional interface which yields epitaxial single crystal thin film growth and then releases the resulting film with little effort.
The models proposed to explain such a process are 1) finite thickness polarization induced interface reorganization (i.e. the polarization catastrophe) 2) double layer Te nucleation, 3) van der Waals induced epitaxy and 4) weak covalent bonding.

Polar oxides (such as sapphire) are known to build up an unstable polarization due to unbalanced charge distribution (dipoles) in their unit cells, resulting in a small by finite voltage.
This unstable polarization must be resolved as it builds to millions of volts over a crystal.
This voltage can be resolved by reorganization of the surfaces to counteract the dipole, or via charge transfer from an overlayer.
Either of these mechanisms could have a impact on the interface between the CdTe and the sapphire.
Both atomic reorganization and charge transfer have the possibility of drastically altering the bonding environment leading to the liftoff phenomenon.

Investigations by this group into the liftoff phenomenon have included the use of density functional theory (DFT) modelling of the energy landscape of Cd and Te atoms on a sapphire surface, and their preferred position.
Modelling of this system has indicated that there may be a preference for two layers of Te to bond to the sapphire substrate before the atoms form the typical CdTe structure.
This two-layer structure immediately at the interface may be beneficial during nucleation however after some critical thickness the CdTe grown above may influence this layer such that the bonding is weakened.

Finally, it is possible for the bonding energies between the initial thin film nucleation layer and substrate to be small, with bonding being more akin to van der Waals forces (physisorption) than to the typical covalent or semi-ionic bonding seen in solids.
The bonding is sufficiently strong enough to provide a good template for epitaxial growth, but when strong perturbative forces (thermal strains, mechanical peeling) are applied, the bonding can be overwhelmed.
This weak bonding regime may be due to the strongly ionic nature of the CdTe semiconductor (\(\sim\)70\% ionicity) combined with the highly stable oxide surface of sapphire.
\section{Implications for Symmetry and Energy at Epitaxial Surfaces}
The interface that lies between CdTe and sapphire has been a place of great interest for the Preston research group for many years.
While sapphire can be described simply using its lattice constants and space group, such a description leaves out many complicating factors.

From a symmetry point of view, the body diagonal of CdTe fits onto sapphire with two geometric orientations.
This coincident surface net, where the two crystals meet, offers the initial template for alignment of the CdTe, promoting a two-phase nucleation and growth.
Such orientation relationships are key for epitaxy to occur, with fewer geometric possibilities being preferred, to reduce grain boundaries.

If growth occurred as in graphoepitaxy, where only geometry defines crystal orientation, this would still lead to defective material.
From an energy point of view, the surface of sapphire can offer an energy landscape which selectively promotes one geometric orientation.
For such preferences to exist, the surface must be stable and uniform, a difficult goal with the multi-layered structure of c-plane sapphire.
When a uniform surface does exist, the added energy landscape is superimposed on the geometry, resulting in a single orientation being preferred for nucleation at the surface.
The role of the surface energy landscape is a key factor in understanding epitaxy on more complicated surfaces such as complex oxides, as this work demonstrates that surface termination can have a strong effect on nucleation.
