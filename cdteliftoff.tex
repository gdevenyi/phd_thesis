\section{Introduction}
The growth of CdTe on oxide substrates, specifically single crystal sapphire, has been a 
region of great interest for the Preston research group. Work has been published on 
optimization of growth parameters, the role of lattice constants, and the optical 
properties of the resulting thin films\cite{svetlana-work,stephen-paper}.

While the work investigating the growth parameters and properties of the resulting thin 
films has progressed to a high level, relatively less investigation has been focused on 
explaining the surprising success that has been achieved with this material system. With 
a relative lattice mismatch of 3.7\% between CdTe and sapphire, a difference in the 
crystallographic space group (cubic CdTe versus hexagonal sapphire), and a vast chemical 
difference (high ionicity semiconductor CdTe versus complex oxide sapphire), the high 
quality single crystal nature of the grown thin films is far from expected.

In this work investigating the CdTe on sapphire heteroepitaxial material system, the 
unexpected high quality growth is examined through the lens of symmetry and energy at the 
epitaxial 
interface. These examinations reveal some explanatory models for the epitaxial alignment 
and defects present in the thin films. These examinations also reveal the previously 
undocumented and highly surprising result that CdTe is not bonded nearly as strongly as 
expected to the sapphire substrates after epitaxial growth, resulting in the 
technologically relevant liftoff phenomenon. This work was completed in collaboration 
with Mr. Stephen M. Jovanovic (growths), Dr. Kristoffer Mienander (DFT, surfaces) and Ms. 
Steffi Woo (TEM).

\section{Background}
CdTe is a cubic semiconductor (a = 6.14xxx) with a very strong propensity to grown 
(111)-up, that is 
alternating layers of cadnimum and telurium, regardless of the structure of the 
underlying substrate. Thus the key requirement for the growth of quality single crystal 
thin films is to control the nucleation and in-plane orientation. \textalpha-Al$_2$O$_3$ 
(sapphire) 
is a rhobehedral complex oxide (a = xxxx, c = yyy), which presents a hexgonal surface net 
on its c-plane surface. As had been previously investigated by the Preston research 
group, the (110) diagonal of CdTe matches to a lattice constant of sapphire to within 
3.5\%.

\section{Experimental}
CdTe thin films were deposited on single crystal c-plane \textalpha-Al$_2$O$_3$ $\pm$ 
0.5\degree wafers, obtained from MTI Crystals Inc and diced into 12 mm $\times$ 12 mm 
squares. Prior to deposition, substrates were solvent cleaned in an ultrasonic bath. 
Samples were loaded into a custom pulsed laser deposition chamber at a base pressure of 
1x10E-7 Torr and in-situ annealed at 450\degree\celsius for 30 minutes. CdTe thin films 
were deposited by pulsed laser deposition using a GSI Lumonics IPEX-848 KrF excimer laser 
with a wavelength of 248 nm. Pulses from the laser were focused and rastered radially 
onto a rotating CdTe 2.54 cm diameter target with a spot size of 4.25 
mm\textsuperscript{2} and average 
energy density of 1.8 J/cm\textsuperscript{2}. The CdTe 5N (99.999\%) pressed powder 
target obtained from 
Princeton Scientific was stoichiometric and undoped. During growth samples were kept at a 
nominal temperature of 300\degree\celsius via a Pt-Rh thermocouple on the growth furnace 
surface. 
Films were grown to a thickness of 100 nm, as determined by optical and stylus 
profilometry.

Structural information was obtained using 2DXRD techniques. A Bruker SMART6000 CCD 
detector on a Bruker 3-circle D8 goniometer with Rigaku RU-200 rotating anode X-ray 
generator and parallel-focusing mirror optics were used for the data collection. 2DXRD 
data was processed into pole figures using Bruker GADDS.
\section{Results and Discussion}



\section{Implications for Symmetry and Energy at Epitaxial Surfaces}
