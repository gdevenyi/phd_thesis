\section{Introduction}
The work presented in this thesis, being experimental in nature, hinges on numerous experimental techniques. The fundamental components of this work rely upon extensive use of X-ray diffraction (XRD) and transmission electron microscopy (TEM) in non-standard and novel ways. These techniques will be examined and examined in some detail, as their novel usage is integral to the work presented here. Several other experimental techniques are also utilized in this work, but they are used in their everyday implementations as seen in many papers in literature. These experimental techniques will be discussed only briefly for brevity. The two primary growth techniques used in this work will also be described, however as this work concentrates on epitaxy is a general phenomenon the intricacies and parameter spaces of these techniques will not be considered again for brevity.

\section{X-Ray}
\subsection{2DXRD - Reciprocal Space Mapping}
\label{sec:2DXRD}
\subsection{High Resolution XRD}

\section{Electron Microscopy}
\section{TEM}
\section{STEM}
\section{SEM}

\section{Growth Techniques}
The work presented in this thesis are prepared primarily by two different growth methods, pulsed laser deposition (PLD) and molecular beam epitaxy (MBE). While these methods are quite distinct in their properties and the regimes under which they operate, the same material systems were not prepared by both systems, so direct comparisons cannot be made. Nevertheless, the differences between these growth processes will be examined in some detail to provide sufficient motivation for their choice for the given experiments.
\subsection{PLD}
\subsection{MBE}
\subsection{Thermal Dewetting}