\section{Introduction}

\section{Background}

\section{Experimental}
The gold nanostructures were formed through the deposition of gold films on MgAl\textsubscript{2}O\textsubscript{4} substrates (MTI Corp.)
followed by an annealing procedure which facilitated film
dewetting and nanostructure formation. The films were
sputter-coated at room temperature to a thickness of 5 \AA-15
\AA with a GATAN PECS Model 682 ion beam coating/etching system. The samples were then placed in a tube
furnace with a 100 sccm flow of argon, heated to 1100\celsius~
in 45 min, and then held at that temperature for 1 h.
Following this treatment, the sample was cooled to 1000\celsius~
in 30 min, held at that temperature for an additional hour,
and then allowed to cool to room temperature over an interval
of approximately 8 h. Holding the temperature at both 1100
and 1000\celsius~was crucial to the formation of the nanostructures described here. Removal of either step results in the formation of faceted gold spheres sitting directly on the substrate.

Scanning electron microscopy (SEM) images of the gold
nanostructures formed on the (100), (111), and (110)
MgAl\textsubscript{2}O\textsubscript{4} substrates, obtained using a JEOL-7000F SEM in
secondary electron mode, are shown in Figure 1. For each
substrate orientation, one observes two types of features, (i)
spheres supported by a necking region attached to a geometrically shaped base (Figure 1a-c) and (ii) standalone base
structures (Figure 1d-f). Convergent beam electron diffraction (CBED) performed using a Phillips CM12 confirmed
that the supported spheres are crystalline. For each case, the shape of the base structure reflects the underlying symmetry
of the substrate which is four-fold, three-fold, and two-fold
symmetric for the (100), (111), and (110) surfaces, respectively. X-ray diffraction measurements, using a Bruker 6000
CCD detector on a Bruker three circle D8 goniometer with
a Rigaku RU-200 rotating anode Cu KR X-ray generator and
parallel-focusing mirror optics, were used to determine the
substrate orientation relative to the edges of the base
structures and are denoted on the three top-down SEM
images (Figure 1d-f).
\section{Results and Discussion}
The crystallographic alignment of
the nanostructures is a clear indication of epitaxy and
is strongly suggestive of {111} gold faceting of the base
structures associated with the [100]- and [111]-oriented
substrates. For the (110) surface, the standalone base
structures are ill-defined and show no obvious faceting, while
those formed in combination with a sphere show shapes
consistent with mixed faceting, possibly having {111} and
{100} facets for the short and long dimension, respectively.
The standalone base dimensions are remarkably uniform with
side lengths of 40, 65, and 65 nm $\times$ 110 nm for the (100),
(111), and (110) substrates, respectively.
\begin{figure}
    \centering
    \includegraphics[width=0.8\textwidth]{nanogold_sem}
    \caption{\label{fig:nanogold_sem}SEM images showing the gold nanostructures formed
        on MgAl\textsubscript{2}O\textsubscript{4} substrates. The three upper images show spheres
        supported by a necking region attached to a geometrically shaped
        base for the (a) [100]-, (b) [111]-, and (c) [110]-oriented substrates.
        Each of these images was taken at a 70\degree tilt. The aura seen around
        nanostructures is an artifact of imaging. The three lower images
        show the top-down view of both standalone base structures and
        supported spheres for the (d) [100]-, (e) [111]-, and (f) [110]-
        oriented substrates. The in-plane Miller indices of the substrate are
        denoted on each of these images. For all cases, the samples were
        coated with a thin layer of platinum to improve imaging. Imaging
        without platinum shows the same structures but is of poor quality
        due to substrate charging effects.}
\end{figure}

While there are two basic types of nanostructures formed
on each substrate orientation, these structures are found in
various stages of development. For the most part, the bases
are well-developed and show little size variation. The
spherical structures, however, vary dramatically both in their
size and position relative to the base structures. Figure 2
shows a series of top-down SEM images for the case of the
(111) MgAl\textsubscript{2}O\textsubscript{4} substrate showing an evolution of the
nanostructures from a standalone triangular base to bases
supporting spheres of increasing size. Notable is the fact that
the nanostructure, shown in Figure 2b, manifests itself as a
small sphere which is offset from the centre of the base while for the larger spheres, this asymmetry disappears. Such
asymmetries are observed for all three substrate orientations,
but only those nanostructures formed on the (111) substrate
consistently show a centrally placed sphere for the large
sphere sizes. Also noteworthy is a systematic effect whereby
the size of the triangular base structure increases for sphere
diameters greater than 80 nm (see Figure 2e and f).
\begin{figure}
    \centering
    \includegraphics[width=0.9\textwidth]{nanogold_progression}
    \caption{\label{fig:nanogold_progression}SEM images showing the top-down view of gold
        nanostructures formed on the (111) surface of MgAl\textsubscript{2}O\textsubscript{4} substrates.
        The sequence of images is chosen to show progressively larger
        spheres atop the base structures. Note that only the smallest sphere,
        shown in (b), is offset from the center of the triangular base and
        that the base size is slightly larger when supporting larger spheres,
        as is the case for (e) and (f). The size of each image is 130$\times$30
        nm\textsuperscript{2}.}
\end{figure}

The base structures formed on each substrate orientation
are a clear consequence of the epitaxial relationship formed
between gold and the latticed-matched MgAl\textsubscript{2}O\textsubscript{4} substrate.
This is apparent from the fact that the geometries of the base
structures mimic the underlying symmetry of the substrates.
It is also likely that the base sizes are a consequence of the
substrate-imposed strains. Arguments based on epitaxy,
however, cannot explain the self-assembly of the gold spheres
atop the base structures and the associated necking behaviour
which facilitates their connection to the base. It is our
hypothesis that the necking behaviour results from an attempt
to minimize the surface energy of the structure. Thus, the
overall shape of these nanostructures is governed by an
interplay between the constraints imposed by epitaxy and a
requirement that the surface free energy be minimized. This
situation has much in common with formation of soap
bubbles affixed to a wire frame\cite{RefWorks:95}. This statement is based
on the facts that the frame imposes a constraint analogous
to that imposed by lattice mismatch and that the shape of
the soap bubble is, to a large degree, determined by the
surface free energy. This analogy provided the impetus for
applying the well-developed models associated with soap
bubble formation to the nanostructures described here. Such
modelling has also been successfully used to predict the
equilibrium shape of biological lipid bilayers (blood cells)
when exposed to abnormal pressure, temperature, magnetic,
and chemical environments\cite{RefWorks:99,RefWorks:102,RefWorks:47,RefWorks:100,RefWorks:101,RefWorks:103}.

The gold nanostructures were modelled as a continuum
elastic surface constrained by a footprint. Three different
footprint geometries (square, equilateral triangle, and rectangle) were used in order to mimic the four-fold, three-fold,
and two-fold symmetries associated with the (100), (111), and (110) surfaces of MgAl\textsubscript{2}O\textsubscript{4}. The size and shape of the
footprint were kept constant during the simulated growths
in order to match the experimental observation indicating a
high degree of base uniformity. For each orientation, the
contact angle (i.e., the angle between the footprint plane and
the tangent plane of any surface connected to the footprint's
edge) was set to a constant where the value was chosen to
be consistent with the observed faceting, as is schematically
shown in Figure 3.

With these constraints, the shape of an open elastic surface
can be fully described by the mean curvature, H, and the
Gaussian curvature, $K$, while its corresponding elastic
properties can be characterized by a bending modulus \textkappa and
a Gaussian modulus \textkappa\textsubscript{G}. The surface energy ($F$) can then be
formulated as
\begin{equation}
F = \int \frac{\kappa}{2}H^2 dA + \int \kappa_G K dA + \lambda S + PV
\end{equation}
where \textlambda, V, S, and dA are the particle's surface tension,
volume, total surface area, and surface area element, respectively\cite{RefWorks:49,RefWorks:97}. The pressure, P, serves as a Lagrange multiplier
which ensures that a constant volume is enclosed between
the structure and the footprint plane. The second term gives
the integrated Gaussian curvature, which is constant according to the Gauss-Bonnet theorem\cite{RefWorks:98}. For a given surface
tension and volume, the equilibrium shape will correspond
to an energy minimum determined by the shape equation
\textdelta F $=$ 0. The structures are more readily solved by first
rescaling the free energy of the model to become dimensionless, such that
\begin{equation}
\tilde{F} = \int \left (\frac{\tilde{\kappa} \tilde{H}^2}{2} + 1 \right)d \tilde{A} + \tilde{P} \tilde{V}
\end{equation}
where
\begin{align*}
\tilde{A} &= A/S_0 & \tilde{V} &= V/S^{3/2}_0 & \tilde{H} &= H S^{1/2}_0 \\
\tilde{\kappa} &= \kappa / \lambda S_0 & \tilde{P} &= P S^{1/2}_0 / \lambda & \tilde{F} &= F / (\lambda S_0)
\end{align*}
and S0 is the area of the base. Thus, according to this
dimensionless free-energy expression, there are two independent controlling parameters,
$\tilde{\kappa}$ and \~{P}.

Helfrich and Ou-Yang\cite{RefWorks:49} have analytically solved the shape
equation for some symmetrical geometries. For the work
presented here, the surface is sectioned into discrete elements
using a triangulation mesh, and a simple dissipative model
is used to minimize the energy, as in eq 2\cite{RefWorks:76}
\begin{equation}
    \frac{\delta \mathbf{r}}{\delta t} = - M \frac{\delta \tilde{F}}{\delta \mathbf{r}}
\end{equation}
where \textbf{r}(t) is the position vector of a point on the particle
surface at the time t and M is a kinetic coefficient. These
methods, developed by Taniguchi et al.\cite{RefWorks:76}, are, however,
unable to simulate large deformations. To circumvent this
limitation, we employed two techniques, equiangulation and
vertex averaging, available through the Surface Evolver software developed by K. Brakke.49 Following the procedure of Lim et al.,39,42,43 the curvature was discretized based on
the methods of Julicher,50 which exactly describe the
curvature in the continuum limit. The variational derivative
used in the triangulation scheme was evaluated analytically.
For the Surface Evolver technique, variables were solved
numerically when analytical variations for discrete curvatures
proved difficult.
\begin{figure}
    \centering
    \includegraphics[width=\textwidth]{nanogold_facets}
    \caption{\label{fig:nanogold_facets}Proposed faceting of the base structures grown on (a) [100]-, (b) [111]-, and (c) [110]-oriented substrates.}
\end{figure}
\begin{figure}
    \centering
    \includegraphics[width=\textwidth]{nanogold_simgrowth}
    \caption{\label{fig:nanogold_simgrowth}Simulations based on the continuum elastic model showing the shape evolution of structures with a triangular footprint as the
        total volume is progressively increased. The shape of these modelled structures is remarkably similar to that of the self-assembled gold
        nanostructures formed on (111) MgAl\textsubscript{2}O\textsubscript{4} substrates (Figures 1b and e and 2). The scale is in arbitrary units.}
\end{figure}

Using the numerical methodology, a progression of
structures with increasing volume were calculated using
square, triangular, and rectangular footprints having contact
angles of 54.7\degree, 70.5\degree, and 35\degree~(long-axis) by 45\degree~(short-axis),
respectively. The chosen contact angles are consistent with
the inferred faceting, as shown in Figure 3. For each case,
the footprint area and surface tension are set to unity, while
the bending modulus is allowed to vary. Figure 4 shows the
calculated progression for the triangular footprint. As the
volume is increased, there is an evolution from a nearly flat
base, to a base with a bulge, and then finally to a spherical
structure supported by a necked region to a triangular base.
These simulated structures are remarkably similar to the gold
nanostructures formed on the (111) MgAl\textsubscript{2}O\textsubscript{4} substrate (see
Figures 1b and c and 2). Similar trends are observed for the
square and rectangular footprints. Figure 5 shows the
simulated high volume structures for the square and rectangular footprints, which, once again, are consistent with
experimental observations. The simulations do not, however,
account for any of the observed asymmetries.
\begin{figure}
    \centering
    \includegraphics[width=\textwidth]{nanogold_square_rect}
    \caption{\label{fig:nanogold_square_rect}Simulations based on the continuum elastic model
        showing the expected high volume shape for the (a) square and
        (b) rectangular (length/width ) 1.42:1) footprints. These structures
        show a resemblance to the self-assembled gold nanostructures
        formed on the (100) and (110) MgAl\textsubscript{2}O\textsubscript{4} substrates (Figure 1) but
        show none of the observed asymmetries. The scale is in arbitrary
        units.}
\end{figure}

The shape of the base structures of the self-assembled gold
nanostructures is strongly influenced by epitaxy and the underlying symmetry of the substrate surface. That being
said, it is difficult to account for the hollowed-out center
observed for the base structure formed on the (100) MgAl\textsubscript{2}O\textsubscript{4}
using epitaxy-based arguments. This feature is, however,
predicted by the continuum elastic model. Figure 6a shows
the topographical color map obtained from the modeling for
the low-volume structure. It clearly shows a circular depres-
sion in the middle of the structure as well as a lobe near
each corner. Motivated by the simulation results, we
examined the topography of the self-assembled gold nanostructure using atomic force microscopy (AFM). Figure 6b
shows a tapping mode AFM image of the base structure
obtained using a Digital Instruments scanning probe microscope (SPM) and a NanoScope IIIa controller. Even though
the features of the nanostructure are somewhat washed out
due to the fact that the resulting image is a convolution of
the nanostructure and the AFM tip geometry, the nanostructure's four lobes and central depression are clearly visible.
\begin{figure}
    \centering
    \includegraphics[width=0.8\textwidth]{nanogold_afm}
    \caption{\label{fig:nanogold_afm}A comparison of the (a) topographical color map derived
        from the continuum elastic model and (b) the AFM image of a
        self-assembled gold nanostructure deposited on the (100) MgAl\textsubscript{2}O\textsubscript{4}
        substrate. Note that the simulation predicts both the presence of
        four lobes at the corners and a central depression. The scale of the
        topographic image is arbitrary, and the scale of the AFM image is
        10 nm.}
\end{figure}

Taken together, the experimental observations and the
continuum elastic modeling results strongly suggest that
epitaxy and a minimization of the surface free energy are
the two primary drivers in determining the shape and size
of the self-assembled gold nanostructures. The fact that
similar structures have not been previously observed is likely
a consequence of the synthesis route used. The route employed here not only involves temperatures well in excess
of those typically used when forming substrate-supported
nanostructures but also requires an annealing profile which
accesses two temperatures. It is possible that the initial 1100\celsius~anneal is needed to induce surface reconstructions in the
MgAl\textsubscript{2}O\textsubscript{4} substrates which are essential to the growth of these
nanostructures. Surface reconstructions have been shown to
strongly influence the growth of both nanostructures\cite{RefWorks:24,RefWorks:16,RefWorks:104}
and thin films\cite{Neretina2009a} when using (100) SrTiO3 substrates. We,
however, consider it more likely that the formation of these
nanostructures requires temperatures in excess of the melting
point of bulk gold (1064\celsius). Significant deviations from
the bulk value, due to finite size effects, are considered
unlikely since such effects appear to occur only in nanostructures smaller than 5 nm\cite{RefWorks:43}. Once molten, the nanostructures are cooled to 1000\celsius~and held at this temperature,
which is just below the melting point. Such a temperature
allows for solidification while maintaining high adatom
mobility. The fact that this fabrication step is crucial to the
formation of the observed intricate nanostructures provides
compelling evidence that considerable adatom motion is
essential. In this scenario, it is likely that an Ostwald-like
ripening process will play a role where larger nanostructures
grow at the expense of smaller ones. Substrate surface steps,
due to the inherent miscut of the substrate\cite{RefWorks:69}, could also
influence the process by providing energetically favorable
nucleation sites. Nanostructure formation is also heavily
reliant on the epitaxial relationship between the nanostructure
base and the underlying substrate. This relationship determines the crystallographic alignment of the base structures,
while the consistency in size is likely a consequence of the
strain imposed by lattice mismatch. Once the 1000\celsius~anneal
ends, the adatom motion will be quickly quenched by the
lower temperatures, allowing no further alterations to the
shape or size of the nanostructures. The end result is
intricately shaped nanostructures whose size and shape are
determined by a minimization of the surface free energy
while simultaneously being subject to the constraints imposed
by epitaxy.

\section{Implications for Symmetry and Energy at Epitaxial Surfaces}