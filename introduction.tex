Epitaxy has been a dominant technological feature since near the very inception of the semiconductor age.
It has also been intimately entwined with the dominant semiconductor up till the present day, silicon.
Silicon has dominated almost every field of semiconductor research for the better part of 40 years, becoming the most well understood material in the world.
This vast focus on silicon has also greatly influenced the thinking of what were at the time fledgling fields, most notably epitaxy.
\emph{Epi-taxis} or ``above -- in an ordered manner'' is the Greek root of the term epitaxy and as it is originally defined, it has been very narrowly interpreted by most of the research field.
Silicon on silicon epitaxy, or homoepitaxy, has been the dominant type of epitaxy both in terms of research and production, due to its relevance to chip manufacturing.
This idea of the `ideal' epitaxy as modelled by silicon homoepitaxy has pervaded the thinking of research into the field, with the material systems being most similar as the result being the most researched, and the most successful.
The material systems most similar to homoeptiaxy (besides other homoepitaxy) are the III-V group semiconductors, specifically the Al/Ga/In-P/As/Sb binary/ternary/quaternary family.
These zincblende semiconductors through the manipulation of exact atomic concentration, can be grown from exactly lattice matched to strongly mismatched. These material systems all have covalent or primarily covalent bonds with strongly preferred atomic sites for the atomic species.
When parameters are optimized, epitaxial growth in such systems is orientationally commensurate with the underlying substrate and dominated by strain effects.

Beyond III-V homologous growth systems, the next most investigated epitaxial process is when these materials are grown on silicon.
In addition to the always-present issue of strain due to the intrinsic lattice mismatch between the III-V family and silicon, the diamond structure of silicon results in an issue of polar (two atom) on non-polar epitaxy\cite{polar-on-non-polar} where group III and group V atoms cannot differentiate between sites on the silicon surface.
Such site non-specificity has been a issue of great interest to the epitaxy research community resulting in numerous attempts (some successful)\cite{polar-on-non-polar-review} to improve growth of such a chemically dissimilar system.

Beyond the extensive research into the epitaxy of silicon and the III-V systems, work into epitaxy has mostly been on a material-by-material basis.
Material systems of interest are examined on a issue by issue basis with the goal of producing high quality material, rather than examining the generalized epitaxy phenomenon.
In this work, the goal has been to expand the understanding of epitaxy through the examination of several model systems, and to develop a more generalized condition for when high quality epitaxy can occur.
Through this work, two key themes were examined in epitaxial systems, the role of symmetry and the role of energy at epitaxial interfaces.
These themes were examined through investigations into three model material systems, III-V semiconductors on silicon, II-VI semiconductors on crystalline oxides, and noble metals on crystalline oxides.
The results from these investigation and the main contribution of this work is a generalization of heteroepitaxy to encompass all material systems where there is a symmetry relationship across the growth interface with an energy landscape which corresponds to that symmetry.

\section{Themes}
\subsection{Symmetry}
The 2D (and as we shall later see sometimes 3D) interface that separates the epitaxial substrate from the growing crystal has a symmetry relationship which relates the substrate to the crystal.
These surface symmetries are different than the bulk symmetry of the substrate and crystal.
In the simplest treatment of surfaces, the surface of a given substrate is simply a slice through the crystal in a given orientation, exposing a plane of atoms which then present a subsymmetry of the bulk crystal.
As will be seen later there are many complications to this model and it is in these complications that we find interesting changes in symmetry, breaks in symmetry and distortions of this 2D surface into a more complex 3D interface.
It is these changes to the surface symmetry, and it's interaction with the growing crystal which have profound and useful implications for epitaxy.
The first major theme investigates the implications of unique interface symmetries, broken symmetries and 3D interfaces on the epitaxial process.

\subsection{Energy}
While symmetry describes the spatial distribution of the landscape presented to a growing crystal, energy describes the magnitude of the effect that landscape has.
Strong energy landscapes cause the symmetry of a given substrate to have it's influence felt strongly, while a weak landscape can have a subtle effect on growth.
The energy landscape of the epitaxial interface can vary over a large range, and the role of these strengths has not been examined extensively.
The second major theme of this work investigates the implications of energy landscapes outside the typical heteroepitaxy regime, specifically relating to the weaker energies.