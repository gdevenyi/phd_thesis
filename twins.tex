\subsection{Introduction}
Since the microelectronics revolution, Silicon has been a dominant material 
for the production of devices for a variety of applications. Silicon has 
non-ideal properties for a variety of applications but remains dominant due 
to its well-understood processing parameters and large manufacture install 
base, providing economies of scale. The III-V semiconductors offer superior 
properties for a variety of applications when compared to silicon, and are 
actively used in applications were performance is valued above other 
considerations, such as military and space sectors. The goal of integrating 
III-V semiconductors into silicon based microelectronics has spawned extensive 
research into the processing involved to grow or otherwise electronically 
attach these materials with minimal defects.

Under auspices of the ARISE Photovoltaics project, the III-V 
semiconductors GaAs, GaSb, AlSb and InP were grown as thin films under a 
variety of conditions on single crystal silicon substrates, in order to 
examine the growth process and electro-optic properties. The role of the 
varied lattice mismatch, growth parameters, and substrate properties were 
examined and several previously undocumented phenomona were examined due to 
the use of 2DXRD, whose benefits were documented in \ref{sec:2DXRD}.

The formation of epitaxial (or growth) twins was found to be a key area where 
the literature had performed little examination. The role of twins in the 
formation of electronic defect networks, and the effects vicinal (offcut) 
substrates had on their formation were thoroughly examined and a 
explanatory model was developed to explain factors that affect their formation 
and provide proposed routes towards their minimization and 
elimination\cite{devenyi_jap}. This work was completed in close collaboration 
with Ms. Steffi Woo, a Ph.D. candidate in Material Science and Engineering and 
McMaster, with the TEM/STEM work performed exclusively by her and all other 
work being collaborative.

\subsection{Background}
Should a section covering the background of III-V on Silicon go here, or go 
into the introductory material at the beginning of the thesis

\subsection{Experimental} \todo{This section is currently copy-pased from 
paper, is this appropriate?}
Semiconductor thin films (GaAs, InP, GaSb, and AlSb) were deposited on nominal 
(001)-oriented ($\pm$0.5$^\circ$) and vicinal Si substrates (offcut 
4.7$^\circ$ ($\pm$0.25$^\circ$) towards [110]) using a SVT Associates 
molecular beam epitaxy (MBE) system. As-received epi-ready wafers were cleaned 
for 1 min in a 4\% HF in deionized (DI) water dip followed by a 30 sec DI 
rinse immediately prior to their insertion into the MBE load-lock. Before film 
deposition, both the nominal and vicinal Si(001) substrates underwent a 15 min 
degassing procedure at 350~$^\circ$C followed by a thermal treatment at 
800~$^\circ$C for up to 5 min, in order to reconstruct the Si surface into 
single domain 
terraces\cite{NeergaardWaltenburg1995,S1991,Sakamoto1986,Pehlke1991}. A small 
number of single steps are expected to remain on vicinal substrates, a higher 
number on nominal substrates because of the larger terrace length. Growth 
conditions followed established 
protocols\cite{Akahane2004,Balakrishnan2006a,Fischer1986} and yielded 
comparable rocking curve full-width half-maximum for the [004] reflection 
using double crystal X-ray diffraction. AlSb thin films were grown to a 
thickness of 550~nm with a 20~nm GaSb capping layer to avoid oxidation. GaAs 
thin films was grown to a thickness of 600~nm and GaSb to a thickness of 
500~nm. InP samples were grown at 470~$^\circ$C with a V/III flux ratio of 2 
at a growth rate of 1 $\mu m$/hr, resulting in a thickness of 600~nm. Double 
crystal X-ray and TEM data also revealed that all films are fully relaxed by a 
network of interfacial misfit dislocations\cite{Vajargah2011}. GaSb samples 
were grown in the presence of a 5 nm AlSb buffer layer, as prescribed by 
Akahane \textit{et al}.\cite{Akahane2004}.

Stereographic pole figures were generated for each sample using 2DXRD 
techniques. A Bruker SMART 6000 CCD detector on a Bruker 3-circle D8 
goniometer (Bruker AXS Inc., Madison, WI) with a Rigaku RU-200 rotating anode 
X-ray generator (Rigaku MSc, The Woodlands, TX) and parallel-focusing 
monochromator optics was used for the data collection. Scans were taken with 
the detector centered on the (111) 2$\theta$ of the material of interest and 
the sample rotated through 360$^\circ$ in 0.5$^\circ$ increments about the 
surface normal of the sample. A 1D integration of all frames was used to 
determine the combined width of the (111) peaks using MAX3D software (McMaster 
University)\cite{Britten2007}. The peak width was then used to integrate (111) 
reflections from all frames, including a background and absorption correction 
for the corresponding material with GADDS (Bruker-AXS) software, resulting in 
a pole figure. Pole intensities were obtained from pole figures using a 
circular integration cursor with a 10 pixel radius which was centered on the 
pole so as to maximize the total intensity captured. All pole intensities were 
corrected for structure factor and frame exposure times.

For each sample two \{110\} TEM cross-sections were prepared, one parallel to 
the [110] miscut direction and the other perpendicular. The specimens were 
prepared by the standard procedure of mechanical polishing, dimpling, and 
ion-milling (4 keV Ar-ions at an incident angle of $\pm$4$^\circ$ using a 
liquid nitrogen cold stage for InP) until perforation. Crystallographic 
information of the epitaxial layer was obtained using diffraction contrast 
imaging with a Philips CM12 conventional transmission electron microscope 
(TEM) operated at 120 kV and equipped with a LaB$_6$ filament. In addition, 
electron diffraction analysis was performed using selected area electron 
diffraction (SAD).

\subsection{Results}