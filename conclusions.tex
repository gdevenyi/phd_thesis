The investigations presented here have provided insight into a number of unique cases of epitaxy.
In this work, two themes outside the usual epitaxial process were examined, unusual symmetry relationships and unusual energy relationships between the host substrate and the epitaxial crystal.
Examinations of symmetry involved epitaxy of systems where the 2D symmetry net is preserved over the interface, while the two crystals themselves have different symmetries or where symmetry is broken across the interface.
Examinations of unusual energy relationships involved polar on non-polar epitaxy of face-centred cubic semiconductors as well as epitaxial growth on oxide substrates.

On the theme of symmetry this work has documented cases where surface reconstructions through steps, a case of symmetry breaking, has resulted in suppression of growth defects, improving epitaxial thin film quality.
The role of the asymmetric (211) surface, again a case of symmetry breaking, has shown that interfacical misfit strain can be accommodated even for largely mismatched systems.
Investigations into the surface reconstructions of oxide substrates resulted in insight into how the step-flow growth mode induced by steps can preferentially nucleate and grow textured epitaxial thin films to improve film quality.
The surface reconstructions of oxide substrates have also been shown to successfully nucleate epitaxial thin films of a different orientation than those nucleated on the bulk substrate, opening up a new field of epitaxial matches with surface reconstructions.

On the theme of energy, this work has demonstrated that epitaxial growth can occur between materials and substrates which are traditionally considered to be weakly bonding. This weak bonding regime was found to be so weak that liftoff of epitaxial thin films and reuse of the resulting substrates was possible. Furthermore, the weak epitaxial regime between noble metals and oxides was explored and demonstrated some novel intricate nanostructure formation in materials which generally considered to be very stable. Finally, investigations into the growth of CdTe nanowires demonstrated a method of surface preparation which can control the nucleation and growth process of these nanowires by modifying surface energies.

The results from these two themes have expanded the conditions under which epitaxial growth can be expected in two key ways, surface reconstructions can provide new surface lattice constants for growth, and weak chemical attraction between materials does not immediately negate good epitaxy. Both these results open up new fields for research and challenge other fields to provide the necessary supporting measurements to advance these ideas further.